\section{Conclusion}

With this paper, we created a critical review
of the systematic mapping study
``A Systematic Mapping Study on Microservices Architecture in DevOps''
by \smsAuthors \wsl.

In \autoref{sec:introduction} the reader was introduced into the
topic and various prerequisites were specified. We introduced
terms like ``Microservice'', ``DevOps'' and methodologies from
design science and empirical software engineering like ``SMS'',
``SLR'', and ``SGLR''.

Within \autoref{sec:summary} the paper gave a brief objective
summary for the reviewed SMS. The methods used are described
with the conducted search and the found results. Afterwards
the reader got an overview of the held discussion in the paper
and the derived research challenges.

The critical review in \autoref{sec:review} then gives the
reader a subjective review with critics to certain topics.
In general, the study covers a broad area of research
and did sum up its results in a clean way.

The reviewed SMS yielded good results in the specific
context of ``DevOps''. However the search terms that were
used did limit the results to publications what contain the
word ``DevOps'' in their title. This excluded results which
may have opinions and solutions to found problems.
Furthermore, the categorical exclusion of gray literature
did limit the result to problems, solutions, and tools from the
academia without the results of the industry. The industry is a key driver
for this topic in computer science and should be
included in such a study. One possible outcome could have been
that more research should be conducted based on statements and
findings from the industry.

As a result of this review, a conclusion could be to conduct
another study with regard to the problems stated in \wsls and
map solutions from gray literature to their problems. Another
possibility is to create further research to topics
presented as possible solutions in \autoref{sec:review} and
create peer-reviewed publications that include gray literature.
