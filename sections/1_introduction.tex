\section{Introduction}
\label{sec:introduction}

This paper is a review of the systematic mapping study
``A Systematic Mapping Study on Microservices Architecture in DevOps'' \cite{waseem:SMSMSADevOps}.
The goal is to introduce the reader to the topic, explain the used
methods in empirical software engineering and give a critical review
of the conducted study and the results.

The study used a broad search over several well-known publication-databases
and searched for research material on the topic of microservice architectures
in the DevOps context. After the first search additional material
was searched for with a technique called ``snowballing''. All this material
was then screened and analyzed and after an initial selection was made,
the studies were fully read and information extracted. The results
are then mapped and categorized according to guidelines.

The study shows various problems and their corresponding solutions
along with some challenges (problems without any purposed solution).
Some of the problems do not have a solution because gray literature is
not allowed in the search.

The remainder of the paper will introduce terms and principles that are used
in the reviewed study, as well as topics that a reader needs for the understanding
of the study and this review. Furthermore, this paper creates an objective summary,
a critical review and a derived conclusion of the reviewed paper \cite{waseem:SMSMSADevOps}.

\subsection{Design Science}

Design Science has the main purpose of achieving knowledge and a general understanding
about a domain. Design Science contains several guidelines
according to Alan R. Hevner (among other authors). Those guidelines are \cite{hevner:DesignScience}:

\begin{itemize}
    \item \textit{Design as an artifact}: Design Science must produce an artifact
    \item \textit{Problem relevance}: The objective is to develop solutions to relevant business problems
    \item \textit{Design evaluation}: The usefulness of an artifact must be demonstrated with evaluation methods
    \item \textit{Research contributions}: It must provide clear and verifiable contributions to the topic
    \item \textit{Research rigor}: The research relies on rigorous methods in construction and evaluation of the model
    \item \textit{Design as a search process}: The search for artifacts requires satisfying laws to be in place
    \item \textit{Communication of research}: The targeted audience should be technology based as well as management based
\end{itemize}

With this guidelines, it is possible to model a domain of interest and acquire
specific knowledge about it.

\subsection{Empirical Software Engineering}

Empirical Software Engineering (ESE) provides a base for discussion and methods for empirical
research regarding software engineering topics. \textit{Empirical} means in the context
of software engineering that various results are taken into consideration. The results
of the methods are proven by existing publications.

\subsubsection{Systematic Mapping Study (SMS)}

A SMS is a defined method to gather, analyze, classify and structure a field of interest.
The analysis focuses on frequency and topics for a
field. It is a defined process in which the following steps take place \cite{petersen:SMS}:

\begin{enumerate}
    \item Define research questions (RQs) and topic
    \item Define search query and parameter
    \item Search for articles and publications in given databases
    \item Analyze and screen the results (i.e. quality assessment and data analysing)
    \item Classify and map the given articles
\end{enumerate}

The result of an SMS allows readers and researchers to determine the coverage of the given
field of interest \cite{petersen:SMS}. In the reviewed paper, the search yielded 47
publications as ``relevant'' for the SMS.

\subsubsection{Systematic Literature Review (SLR)}

An SLR is a method of ESE to systematically analyze and review a given topic. It uses
methods to collect secondary data and critically reviews the given research study.
The search for additional data can involve published as well as unpublished work
on the subject \cite{siddaway:SLR}. SLR and SMS both enable researcher to acquire
knowledge about a topic. Whereas the SMS does this on a wide scale, the SLR
goes into depth of a domain of interest. Babak Farshchian and Yngve Dahl described
the difference as: ``An SMS uses the same basic methodology for searching and 
analyzing literature as in a SLR. An SMS, on the other hand, aims at
creating a map of a wide research field. \dots The knowledge created by an SMS can be
used as the basis for further research (Kitchenham et al 2011),
for instance, as a pre-study for one or several
SLRs in specific areas.'' \cite{farshchian:SMSvsSLR}.

\subsubsection{Systematic Gray Literature Review (SGLR)}

A SGLR is essentially the same as an SLR with a very important difference:
It does not only consider published and unpublished \textit{peer reviewed} work,
but also ``Gray Literature''. Gray literature is evidence and material that
is not published in commercial and peer reviewed publications \cite{paez:GrayLiterature}.
In the context of computer science, gray literature can provide important statements
and evidence towards topics that are more driven by businesses than by the academia.
Commercial companies drive the innovation around computer science nowadays. 
Those companies need solutions for acute problems and therefore do not wait on papers and evidence
to be peer-reviewed until they move on.

\subsection{Microservices and DevOps}

Since the topic of the reviewed paper does conduct an SMS over ``Microservices in DevOps'',
it is utterly important to define those terms so that any reader of this paper
understands the base of the terms on which the conclusions are built upon.

\subsubsection{Microservices}

Microservices (sometimes referred to as ``Microservice Architecture'') is an application
structural style. The style focuses on building several small services that cooperate
together to create an application. Those services are often deployed on distributed
systems. Microservices adhere to the following tenets \cite{zio:MSATenets}:

\begin{itemize}
    \item \textit{Fine-grained interfaces}: The work unit encapsulates logic
    around a single topic of work and exposes the interface remotely
    \item \textit{Domain-Drive Design (DDD)}: The services are conceptional
    created around business-driven development patterns
    \item \textit{IDEAL}: \textbf{I}solated State, \textbf{D}istribution,
    \textbf{E}lasticity, \textbf{A}utomated Management and \textbf{L}oose Coupling
    \item \textit{Polyglot Persistence}: Multiple programming paradigms (e.g.
    object-oriented and functional) and storage paradigms (e.g. NoSQL and relational
    database systems) are used in a polyglot programming and persistence strategy
    \item \textit{Lightweight Containers}: The work units are containerized and
    deployed via corresponding channels (e.g. Docker)
    \item \textit{Automated Continuous Delivery}: During service development,
    a high degree of automation is used to deploy the work unit
    \item \textit{Lean Holistic Management (DevOps)}: Largely automated practices
    are used to tackle configuration, performance, monitoring and fault management
\end{itemize}

These tenets define the work units of a microservice architecture. This makes
a MSA a highly flexible and dynamic approach to develop software \cite{zio:MSATenets}.

\subsubsection{DevOps}

The term ``DevOps'' is a mash-up between ``Development'' and ``Operations''
which are two strong and important terms in modern software engineering.
DevOps provides a set of practices with the intend to reduce the time
of a change to the code to the production environment
while maintaining a high quality of the software \cite{bass:devops}. As an example, when a change
is committed to the version control system (VCS), like GitHub, the build pipeline
will test and build the container automatically. Afterwards, the container
is deployed to a test environment and on acknowledgement, the container is
then shipped to production.
